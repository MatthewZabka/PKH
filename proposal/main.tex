\documentclass[11pt]{article}

\usepackage{amsmath, amssymb, amsthm}
\usepackage{graphicx}   % For figures
\usepackage{cite}       % For compressed citations
\usepackage{hyperref}
\usepackage[margin=1in]{geometry}

\begin{document}

\begin{center}
    {\Large \textbf{Persistent Khovanov Homology for Knots and Banded Knots}}\\[1.5em]

    {\large Matthew Zabka}\\
    Department of Mathematics and Computer Science, Southwest Minnesota State University\\
    Email: \texttt{matthew.zabka@smsu.edu}\\[1em]

    {\large Anzor Beridze}\\
    School of Mathematics and Computer Science, Kutaisi International University\\
    Email: \texttt{anzor.beridze@kiu.edu.ge}\\[2em]

    {\large \textbf{Draft Proposal for Internal and Interdisciplinary Discussion}}\\[0.5em]
    (This document is a preliminary working draft intended for conceptual development and collaborative review.
    Additional authors may be added as the project evolves.)
\end{center}

\vspace{2em}

% Create a new 1st level heading
\section*{Introduction}

Knot theory has emerged as a powerful framework for understanding the geometry and topology of biological macromolecules. 
DNA often becomes knotted or linked during replication, transcription, and packaging, and cells rely on specialized 
enzymes such as topoisomerases to regulate this topological complexity~\cite{Stasiak2009DNAunlinking,Bates2022Topoisomerases}. 
Proteins, too, can display deeply knotted or slipknotted configurations whose folding pathways and functional implications 
remain the subject of active investigation~\cite{Faisca2015KnottedProteins}. 
At larger scales, chromatin and chromosomes exhibit intricate polymeric entanglements that influence their spatial 
organization and epigenetic state~\cite{Smrek2017KnottedChromatin}. 
Even RNA, although generally thought to avoid deep knots, can exhibit nontrivial entanglements and topological 
features in three-dimensional structural models~\cite{Boniecki2021RNAEntanglement}. 
These examples illustrate that knot theory---including both classical knot invariants and newer homological tools---is 
well positioned to contribute to the quantitative characterization of complex, noisy, or partially observed biological structures.

Despite this widespread appearance of knotted and entangled structures in biology, in many experimental settings the 
topology of a macromolecule cannot be determined exactly. Cryo--electron microscopy, for example, often produces 
three--dimensional density maps with unresolved or ambiguous loop regions, leaving substantial uncertainty about the 
underlying geometric configuration~\cite{Henderson2013CryoEM}. Chromatin conformation capture techniques such as Hi--C 
provide only pairwise contact frequencies rather than full spatial embeddings, making it impossible to determine a 
definitive knot or link type from the data~\cite{Dekker2013ChromatinHiC}. Even when explicit polymer models are used to 
reconstruct chromosome geometry, the resulting embeddings may display topological features---including knots, links, 
and slipknots---that fluctuate across model realizations or remain only partially localized~\cite{Benedetti2014ChromosomeTopology,
Millett2005KnotLocalization}. These challenges highlight the need for mathematical tools capable of quantifying 
topological structure in settings where only incomplete, noisy, or ensemble-based information is available. Persistent 
homology provides a natural framework for addressing such uncertainty, allowing knot-like features to be detected, 
tracked, and compared across varying spatial scales or model parameters.

In addition to uncertainty arising from incomplete data, many biological processes modify molecular topology through 
operations that are mathematically equivalent to attaching or removing bands. Enzymes such as topoisomerases and 
site--specific recombinases alter DNA configuration by cutting, twisting, and rejoining strands, producing controlled 
topological changes that closely parallel band surgeries in knot theory~\cite{Wang1996TopoReview,Crisona1999RecombinationCell}. 
Such reconnection events can generate or remove crossings, change knot or link type, or create composite structures 
from simpler precursors, and therefore represent natural biological analogues of band attachments. More broadly, the 
dynamic organization of chromatin and other polymeric biomolecules involves local strand--passage and loop--formation 
events that likewise resemble band moves at a coarse topological level~\cite{Stasiak1988Science}. Despite the ubiquity 
of these band--like transformations in molecular biology, the mathematical theory of \emph{banded knots} has not been 
systematically developed in this context. This gap suggests an opportunity to integrate band surgery with 
topological data analysis, enabling new tools for detecting, classifying, and quantifying biologically relevant 
topological modifications in settings where experimental resolution is limited.

%%%%%%%%%%%%%%%%%%%%%%%%%%%%%%%%%%%%%%%%%%%%%

\section*{Research Objectives and Proposed Approach}

The goal of this project is to develop a new mathematical and computational framework for analyzing knot-like
structures in settings where both topological uncertainty and biologically meaningful modifications must be taken into
account. Building on the foundations of topological data analysis~\cite{Carlsson2009TopologyAndData} and the algebraic
theory of multiparameter persistence modules~\cite{CarlssonZomorodian2009}, we propose to construct a bi\-parameter
persistence model in which a geometric ``distance'' parameter identifies the underlying knotted core of a structure,
while an independent ``density'' parameter reveals additional strands, attachments, or band--like features that may
arise from enzymatic activity or incomplete structural reconstruction. This bifiltration will serve as the basis for
a theory of \emph{persistent Khovanov homology}, extending both the classical Khovanov homology of links
\cite{Khovanov2000} and its behavior under controlled knot operations~\cite{Turner2006Khovanov} into a parameterized and
data-driven setting. By drawing on the stability principles that underpin persistent homology~\cite{CohenSteiner2007Stability},
our objective is to establish a framework in which Khovanov-theoretic information can be tracked across spatial scales
and through biologically motivated band surgeries, enabling the detection and classification of knot-like configurations
even when full geometric information is unavailable or when the topology itself may change.

To develop this framework, we introduce a distance--density bifiltration that captures two complementary aspects of 
knot-like structures. The first parameter, \emph{distance}, governs a filtration in which points or segments of an 
embedded curve are incorporated according to their geometric proximity, allowing a simplified knotted core to emerge 
across scales. The second parameter, which we refer to as a \emph{density} parameter, governs the addition of 
band attachments between arcs that lie sufficiently close in the ambient space. At low density levels, the filtration 
contains only the most reliably supported portions of the structure, while higher density thresholds allow 
geometrically proximate arcs to be joined by canonical band moves. Such a band move corresponds to a saddle 
cobordism, and therefore induces a well-defined map (up to sign) on Khovanov homology by the functoriality 
results of Jacobsson~\cite{Jacobsson2004}. In this way, the density parameter generates a structured family of knots 
and banded knots connected by cobordism-induced maps, providing a natural setting in which to examine how 
Khovanov homology evolves along a bifiltration. This conceptual foundation leads directly to the research aims 
outlined below.

\subsection*{Specific Aims}
The development of persistent Khovanov homology for knots and banded knots requires advances in both the 
theoretical foundations of bifiltrations and the algebraic structure of Khovanov-type invariants. 
Guided by the distance--density framework described above, our project will pursue the following specific aims:

\begin{enumerate}
    \item \textbf{Formalize the distance--density bifiltration for knots and banded knots.} 
    We will develop a rigorous mathematical model in which geometric proximity and band attachments are encoded 
    as a two-parameter family of inclusions. This includes identifying conditions under which band additions 
    arise canonically from density thresholds and analyzing how these operations interact with classical knot 
    invariants.
    
    \item \textbf{Construct persistent Khovanov homology along the bifiltration.}
    Using the functoriality of Khovanov homology under saddle cobordisms~\cite{Jacobsson2004}, we will define 
    homomorphisms associated to band moves and assemble these into a multiparameter persistence module. Our goal 
    is to characterize the resulting algebraic structures, study their dependence on filtration parameters, and 
    identify features that persist across scales or through band attachments.

    \item \textbf{Investigate computational and structural properties of the resulting invariants.}
    We will implement computational experiments to explore how persistent Khovanov homology behaves for families of 
    knots and banded knots generated by the bifiltration. This includes analyzing small examples, examining how 
    persistent features correspond to known topological operations, and identifying patterns or conjectures that 
    may guide future theoretical development.
\end{enumerate}

These aims establish a coherent pathway toward a persistent version of Khovanov homology adapted to 
settings in which topological modifications arise naturally or in which partial or uncertain structural information 
must be incorporated. Together, they lay the groundwork for a broader theory of categorified invariants in 
multifiltration contexts.


%%%%%% Bibliography %%%%%%%%

\bibliographystyle{IEEEtran}
\bibliography{master}

\end{document}